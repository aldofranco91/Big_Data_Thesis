\documentclass[10pt]{beamer}
\usetheme{Boadilla}
\usepackage[utf8]{inputenc}
\usepackage{amsmath}
\usepackage{amsfonts}
\usepackage{multicol}
\usepackage{listings}
\usepackage{sansmathaccent}
\pdfmapfile{+sansmathaccent.map}
\usepackage{hyperref}
\usepackage{amssymb}
\DeclareGraphicsExtensions{,.pdf,.png,.jpg,.gif}
\usepackage{graphicx}
\setbeamercovered{transparent} 

%\setbeamertemplate{navigation symbols}{} 
%\logo{} 
%\institute{} 
%\date{} 
%\subject{} 

\title[Forecasting Spanish electricity prices] %optional
{\huge {Forecasting Spanish electricity prices}}
 
\author[Universidad Carlos III de Madrid] % (optional, for multiple authors)
{\Large {Aldo Franco Comas}}

\institute[UC3M] 

\titlegraphic{\includegraphics[width=1.5cm]{imagenes/logo_uc3m.jpg}}
 
\date[March, 2018] % (optional)
{Master in Big Data Analytics\\
2017-2018.}
 
\begin{document}

\begin{frame}
\titlepage
\end{frame}

\begin{frame}{Motivation and importance.}
\begin{block}{What did we do?}
\begin{itemize}
\item Forecast to short, medium and long term the electricity prices in Spain.
\item Using SVM, ARIMA models, TBATS and dynamic factor models.
\item Create an App where can be visualized and downloaded these predictions.
\end{itemize}
\end{block}

\begin{block}{Why did we do it?}
\begin{itemize}
\item It is very useful for portfolio managers, producers and utilities companies.
\item Forecasting in real time.
\item No user intervention.
\end{itemize}
\end{block}
\end{frame}


\begin{frame}{Law of supply and demand.}
\begin{figure}[H] 
\centering
\includegraphics[scale=0.2]{imagenes/SD.jpg}
\caption{The Law of Supply and Demand.}
\end{figure}
\end{frame}

\begin{frame}{Similar researches.}
\only<1-3>{
\begin{block}{Short term.}
\begin{itemize}
\item ARMA models in the German EEX market.
\item AR models in the California market.
\item ARMAX models in the PJM market.
\end{itemize}
\end{block}
}

\only<2-3>{
\begin{block}{Medium term.}
\begin{itemize}
\item VAR models in the Iberian market.
\item Forward price models in the Nordic markets.
\end{itemize}
\end{block}
}

\only<3>{
\begin{block}{Long term.}
\begin{itemize}
\item ANN in the Spanish market.
\end{itemize}
\end{block}
}

\end{frame}

\begin{frame}{Software tools}

\begin{columns}
\begin{column}{.5\textwidth}
\begin{figure}
\includegraphics[scale=0.3]{imagenes/R.png}
\end{figure}
\end{column}

\begin{column}{.5\textwidth}
\begin{figure}
\includegraphics[scale=0.1]{imagenes/AWS.png}
\end{figure}
\end{column}
\end{columns}

\begin{block}{R's packages}
\begin{itemize}
\item jsnolite
\item httr
\item lubridate
\item MLmetrics
\item forecast
\item caret
\item shiny
\item cronR
\end{itemize}
\end{block}

\end{frame}

\begin{frame}{How can we download the data?}

REE has developed an information system known as 
\href{https://www.esios.ree.es/es}{E-SIOS}. Where we can access to non confidential information through the ESIOS-API.

\begin{table}[H]
\centering
\caption{Variables downloaded from E-SIOS website.}
\label{my-label}
\begin{tabular}{|c|c|c|c|c|}
\hline
Variable                          & Id  & By Month & By Day & By Hour \\ \hline
Price                             & 600 & Yes      & Yes    & Yes     \\ \hline
Demand Forecast                   & 460 & No       & No     & Yes     \\ \hline
Wind Power Generation Forecast    & 541 & No       & No     & Yes     \\ \hline
Solar PV Generation Forecast      & 542 & No       & No     & Yes     \\ \hline
Solar Thermal Generation Forecast & 543 & No       & No     & Yes     \\ \hline
\end{tabular}
\end{table}

\end{frame}

\begin{frame}{Methodology in order to select the best models.}
\begin{figure}[H] 
\centering
\includegraphics[scale=0.4]{imagenes/TrainTest.jpg}
\caption{Illustration of the rolling forecasting origin procedure.}
\end{figure}
\end{frame}

\begin{frame}{Methodology in order to select the best models.}
\begin{block}{Forecasting Accuracy Measure}
\begin{itemize}
\item $MAPE=\frac{100}{n}\sum_{t=1}^{n}\left | \frac{y_t-\widehat{y_t}}{y_t} \right |$
\item $MAE=\frac{1}{n}\sum_{t=1}^{n}\left | y_t-\widehat{y_t} \right | $
\item $MDAE=median(\left | y_t-\widehat{y_t} \right |)$
\end{itemize}
\end{block}
where $y_t$ is the real price and $\widehat{y_t}$ is the corresponding forecast

\end{frame}


\begin{frame}{Forecasting 24 hours.}
\only<1>{
\begin{block}{Regression Formula}
\begin{align*}
Price \sim & Daily Peninsular Demand Forecast  \\
           & + Peninsular Wind Power Generation Forecast  \\
           & + Solar PV Generation Forecast  \\
           & + Solar Thermal Forecast.
\end{align*}
\end{block}

\begin{block}{Period under study.}
From January 1st, 2016 to June 30th, 2017
\end{block}
}

\only<2>{
\begin{block}{Algorithms for forecast 24 hours}
\begin{multicols}{2}
\begin{itemize}
\item SVM with linear kernel.
\item SVM with polynomial kernel.
\item SVM with radial kernel.
\item KNN.
\item Gaussian Process with polynomial kernel.
\item Gaussian Process with radial kernel.
\item Dynamic Factor Model.
\end{itemize}
\end{multicols}
\end{block}

\begin{block}{Number of days for train.}
\begin{itemize}
\item 7, 14, 21, 42, 84 days.
\item 100, 200, 300, 400 and 500 days for DFM.
\end{itemize}
\end{block}
}

\only<3>{
\begin{block}{Hyperparameter tuning}
\begin{itemize}
\item SVM with linear kernel $\Rightarrow$ C.
\item SVM with polynomial kernel $\Rightarrow$ degree, scale, C.
\item SVM with radial kernel $\Rightarrow$  sigma, C.
\item Gaussian Process with polynomial kernel $\Rightarrow$ degree, scale.
\item Gaussian Process with radial kernel $\Rightarrow$ sigma .
\item KNN $\Rightarrow$ kmax, distance, kernel. 
\end{itemize}
\end{block}
}

\only<4>{
\begin{block}{Algorithms for forecast 24 hours}
\begin{multicols}{2}
\begin{itemize}
\item {\color{red}SVM with linear kernel.}
\item SVM with polynomial kernel.
\item SVM with radial kernel.
\item KNN.
\item Gaussian Process with polynomial kernel.
\item Gaussian Process with radial kernel.
\item Dynamic Factor Model.
\end{itemize}
\end{multicols}
\end{block}

\begin{block}{Number of days for train.}
\begin{itemize}
\item {\color{red}7}, 14, 21, 42, 84 days.
\item 100, 200, 300, 400 and 500 days for DFM.
\end{itemize}
\end{block}

\begin{table}[H]
\centering
\begin{tabular}{|c|c|c|c|l|l|l|}
\hline
Measure & Min  & Q1   & Median & Mean & Q3   & Max   \\ \hline
MAE     & 0.57 & 2.13 & 2.94   & 3.39 & 3.96 & 18.26 \\ \hline
\end{tabular}
\end{table}

}

\end{frame}

\begin{frame}
\begin{figure}[H]
\centering
\includegraphics[scale=0.4]{imagenes/SVM_Good.eps}
\caption{Real and predicted prices when a low MAE is obtained using SVM with linear kernel for the next 24 hours.}
\end{figure}

\end{frame}

\begin{frame}
\begin{figure}[H]
\centering
\includegraphics[scale=0.4]{imagenes/MAE_24.png}
\caption{Daily MAE for 1-24 Hours from January 1st, 2016 to June 30th, 2017.}
\end{figure}
\end{frame}

\begin{frame}{Forecasting 144 hours.}
\only<1>{
\begin{block}{Formula regression}
\begin{align*}
Price \sim & Daily Peninsular Demand Forecast 
\end{align*}
\end{block}

\begin{block}{Period under study.}
From January 1st, 2016 to June 30th, 2017
\end{block}
}

\only<2>{
\begin{block}{Algorithms for forecast 144 hours}
\begin{itemize}
\item Dynamic Factor Model with 2 factors.
\item Dynamic Factor Model with 3 factors.
\item DFM with 2 factors + Linear Regression.
\item DFM with 3 factors + Linear Regression. \end{itemize}
\end{block}
\begin{block}{Number of days for train.}
\begin{itemize}
\item 100, 200, 300, 400 and 500 days.
\end{itemize}
\end{block}
}

\only<3>{
\begin{block}{Algorithms for forecast 144 hours}
\begin{itemize}
\item SVM with linear kernel.
\item Dynamic Factor Model with 2 factors.
\item Dynamic Factor Model with 3 factors.
\item DFM with 2 factors + Linear Regression.
\item {\color{red}DFM with 3 factors + Linear Regression. }
\end{itemize}
\end{block}

\begin{block}{Number of days for train.}
\begin{itemize}
\item 100, 200, 300, 400 and {\color{red}500} days.
\end{itemize}
\end{block}

\begin{table}[H]
\centering
\begin{tabular}{|c|c|c|c|l|l|l|}
\hline
Measure & Min  & Q1   & Median & Mean & Q3   & Max   \\ \hline
MAE     & 1.98 & 3.63 & 4.92 & 5.72 & 6.65 & 20.59 \\ \hline
\end{tabular}
\end{table}

}
\end{frame}

\begin{frame}
\begin{figure}[H]
\centering
\includegraphics[scale=0.4]{imagenes/144_Good.eps}
\caption{Real and predicted prices when a low MAE is obtained using LR+DFM with r=3 for the next 144 hours.}
\label{Good_144}
\end{figure}
\end{frame}

\begin{frame}
\begin{figure}[H]
\centering
\includegraphics[scale=0.4]{imagenes/MAE_144.png}
\caption{Daily MAE for 1-144 Hours from January 1st, 2016 to June 30th, 2017.}
\end{figure}
\end{frame}

\begin{frame}{Forecasting 30 days.}
\only<1>{
\begin{block}{Algorithms for forecast 30 days}
\begin{itemize}
\item $ARIMA (p,1,q)(P,1,Q)_7$
\item $TBATS_7$
\end{itemize}
\end{block}
\begin{block}{Number of days for train.}
\begin{itemize}
\item 100, 20, 300, 400 and 500 days.
\end{itemize}
\end{block}
}

\only<2>{
\begin{block}{Algorithms for forecast 30 days}
\begin{itemize}
\item $ARIMA (p,1,q)(P,1,Q)_7$
\item {\color{red}$TBATS_7$}
\end{itemize}
\end{block}
\begin{block}{Number of days for train.}
\begin{itemize}
\item 100, 20, 300, 400 and {\color{red}500} days.
\end{itemize}
\end{block}

\begin{table}[H]
\centering
\begin{tabular}{|c|c|c|c|l|l|l|}
\hline
Measure & Min  & Q1   & Median & Mean & Q3   & Max   \\ \hline
MAE     & 2.18 & 4.14 & 5.39 & 6.75 & 8.41 & 25.33 \\ \hline
\end{tabular}
\end{table}
}
\end{frame}

\begin{frame}
\begin{figure}[H]
\centering
\includegraphics[scale=0.4]{imagenes/Good_1_30.eps}
\caption{Real and predicted prices when a low MAE is obtained using $TBATS_7$ for the next 30 days.}
\label{Good_30}
\end{figure}
\end{frame}

\begin{frame}
\begin{figure}[H]
\centering
\includegraphics[scale=0.4]{imagenes/MAE_30.png}
\caption{Daily MAE for 1-30 Days from January 1st, 2016 to June 30th, 2017.}
\end{figure}
\end{frame}

\begin{frame}{Forecasting 12 months.}
\only<1>{
\begin{block}{Algorithms for forecast 12 months}
\begin{itemize}
\item $ARIMA (p,1,q)(P,1,Q)_{12}$
\item $ARIMA (p,1,q)(P,1,Q)_{12}$ with automatic BoxCox transformation.
\end{itemize}
\end{block}

\begin{block}{Number of days for train.}
\begin{itemize}
\item 60, 80, 100, 120 and 140 months.
\end{itemize}
\end{block}
}

\only<2>{
\begin{block}{Algorithms for forecast 12 months}
\begin{itemize}
\item $ARIMA (p,1,q)(P,1,Q)_{12}$
\item {\color{red}$ARIMA (p,1,q)(P,1,Q)_{12}$ with automatic BoxCox transformation.}
\end{itemize}

\end{block}

\begin{block}{Number of days for train.}
\begin{itemize}
\item 60, 80, 100, 120 and {\color{red}140} months.
\end{itemize}
\end{block}

\begin{table}[H]
\centering
\begin{tabular}{|c|c|c|c|l|l|l|}
\hline
Measure & Min  & Q1   & Median & Mean & Q3   & Max   \\ \hline
MAE     & 2.53 & 5.36 & 8.23 & 9.22 & 10.90 & 30.00 \\ \hline
\end{tabular}
\end{table}


}
\end{frame}


\begin{frame}
\begin{figure}[H]
\centering
\includegraphics[scale=0.4]{imagenes/Good_12.eps}
\caption{Real and predicted prices when a low MAE is obtained using $ARIMA$ for the next 12 months.}
\label{Good_12}
\end{figure}
\end{frame}

\begin{frame}
\begin{figure}[H]
\centering
\includegraphics[scale=0.4]{imagenes/MAE_12.png}
\caption{Monthly MAE for 1-12 Months from January, 2010 to June, 2017.}
\end{figure}
\end{frame}

\begin{frame}{Automatization and Visualization}
\begin{block}{AWS and cron}
\begin{itemize}
\item AWS instance with RStudio and Julia server.
\item cronR package.
\end{itemize}
\end{block}


\begin{alertblock}{Appication}
\href{http://35.157.17.182/shiny/rstudio/sample-apps/Forecast/}{\beamergotobutton{App}}
\end{alertblock}
\end{frame}

\begin{frame}
\begin{center}
{\Huge THE END!!!!}
\end{center}
\end{frame}

\end{document}







